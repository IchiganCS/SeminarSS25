\documentclass[]{rptuseminar}

% Specify that the source file has UTF8 encoding
\usepackage[utf8]{inputenc}
% Set up the document font; font encoding (here T1) has to fit the used font.
\usepackage[T1]{fontenc}
\usepackage{lmodern}

% Load language spec
\usepackage[american]{babel}
% German article --> ngerman (n for »neue deutsche Rechtschreibung«)
% British English --> english

% For bibliography and \cite
\usepackage{cite}

% AMS extensions for math typesetting
\usepackage[intlimits]{mathtools}
\usepackage{amssymb}
% ... there are many more ...
\usepackage{bbold}
\usepackage{bussproofs}
\usepackage{wrapfig}
\usepackage{color}
\usepackage{transparent}
\definecolor{candyapplered}{rgb}{1.0, 0.03, 0.0}

% Load \todo command for notes
\usepackage{todonotes}
% Sebastian's favorite command for large inline todonotes
% Caveat: does not work well with \listoftodos
\newcommand\todoin[2][]{\todo[inline, caption={2do}, #1]{
		\begin{minipage}{\linewidth-1em}\noindent\relax#2\end{minipage}}}

% Load \includegraphics command for including pictures (pdf or png highly recommended)
\usepackage{graphicx}

% Typeset source/pseudo code
\usepackage{listings}

% Load TikZ library for creating graphics
% Using the PGF/TikZ manual and/or tex.stackexchange.com is highly adviced.
\usepackage{tikz}
% Load tikz libraries needed below (see the manual for a full list)
\usetikzlibrary{automata,positioning}

% Load \url command for easier hyperlinks without special link text
\usepackage{url}

% Load support for links in pdfs
\usepackage{hyperref}

% Defines default styling for code listings
\definecolor{gray_ulisses}{gray}{0.55}
\definecolor{green_ulises}{rgb}{0.2,0.75,0}
\lstset{%
  columns=flexible,
  keepspaces=true,
  tabsize=3,
  basicstyle={\fontfamily{tx}\ttfamily\small},
  stringstyle=\color{green_ulises},
  commentstyle=\color{gray_ulisses},
  identifierstyle=\slshape{},
  keywordstyle=\bfseries,
  numberstyle=\small\color{gray_ulisses},
  numberblanklines=false,
  inputencoding={utf8},
  belowskip=-1mm,
  escapeinside={//*}{\^^M} % Allow to set labels and the like in comments
}

% Defines a custom environment for indented shell commands
\newenvironment{displayshellcommand}{%
	\begin{quote}%
	\ttfamily%
}{%
	\end{quote}%
}

%%%%%%%%%%%%%%%%%%%%%%%%%%%%%%%%%%%%%%%%%%%%%%%%%%%%%%%%%%%%%%%%%%%%%%%%%%%%%%%

\title{Mailbox Types}
\event{Seminar: Scalable Distributed Systems in Summer term 2025}
\author{Samuel Klaaßen
  \institute{RPTU Kaiserslautern-Landau, AG Software Technology: Programming Distributed Systems}}

%%%%%%%%%%%%%%%%%%%%%%%%%%%%%%%%%%%%%%%%%%%%%%%%%%%%%%%%%%%%%%%%%%%%%%%%%%%%%%%
\begin{document}
%%%%%%%%%%%%%%%%%%%%%%%%%%%%%%%%%%%%%%%%%%%%%%%%%%%%%%%%%%%%%%%%%%%%%%%%%%%%%%%

\maketitle

%%%%%%%%%%%%%%%%%%%%%%%%%%%%%%%%%%%%%%%%%%%%%%%%%%%%%%%%%%%%%%%%%%%%%%%%%%%%%%%


\begin{abstract}
    Give an abstract of your paper.
    \begin{itemize}
        \item What is the problem area and topic and question you're trying to address?
        \item Why is it a relevant problem ?
        \item How is it solved?
        \item Which results were obtained?
    \end{itemize}
\end{abstract}

%%%%%%%%%%%%%%%%%%%%%%%%%%%%%%%%%%%%%%%%%%%%%%%%%%%%%%%%%%%%%%%%%%%%%%%%%%%%%%

\section{Introduction}
\label{sec:introduction}

    The introduction includes the motivation for the presented work.
    In general, the author will explain the addressed problem and why it is of interest to solve that problem.

\section{Background}
\label{sec:background}
    What needs to be known?

\section{Topic-specific content}
\label{sec:contentSummary}

\section{Discussion}
\label{sec:discussion}
    Don\'t forget your Research Question
    
\section{Conclusion}
\label{sec:conclusions}
    


\begin{figure}
\begin{lstlisting}[language=Scala,numbers=left]
def sort[T](list: List[T])(implicit ord: Ordering[T]): List[T] = {
  list match {
    case Nil => Nil
    case x :: xs =>
      // partition list based on pivot element x
      val (lo, hi) = xs.partition(ord.lt(_, x))  //*\label{srcline:algcmp}
      sort(lo) ++ (x :: sort(hi))
  }
}
\end{lstlisting}
\caption{An example algorithm. Do you recognize it? Line~\ref{srcline:algcmp} is crucial!}
\label{alg:example}
\end{figure}



%%%%%%%%%%%%%%%%%%%%%%%%%%%%%%%%%%%%%%%%%%%%%%%%%%%%%%%%%%%%%%%%%%%%%%%%%%%%%%%
\newpage
\nocite{*}
\bibliographystyle{eptcs}
\bibliography{references}

%%%%%%%%%%%%%%%%%%%%%%%%%%%%%%%%%%%%%%%%%%%%%%%%%%%%%%%%%%%%%%%%%%%%%%%%%%%%%%%
\end{document}
%%%%%%%%%%%%%%%%%%%%%%%%%%%%%%%%%%%%%%%%%%%%%%%%%%%%%%%%%%%%%%%%%%%%%%%%%%%%%%%